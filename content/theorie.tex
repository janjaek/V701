\section{Theorie}
\label{sec:Theorie}
In diesem Versuch wird die Reichweite von $\alpha$-Strahlung in Luft untersucht.
Diese hängt mit der Energie der Strahlung zusammen, da die $\alpha$-Teilchen beim Durchlaufen von Materie über verschiedene Wechselwirkungen ihre diese abgeben.
Einerseits besteht die Möglichkeit, dass die Teilchen ihre Energie durch elastische Stöße bzw. Rutherford-Streuung abgeben, jedoch ist dies nicht der hauptsächliche Grund für die Energieabgabe.
Wesentlich bedeutender ist die Ionisation der Moleküle oder auch die Anregung und Aufteilung dieser. Diese Effekte lassen sich für hohe Energien über die Bethe-Bloch-Gleichung beschreiben:
\begin{equation}
	\label{bbgl}
	\symup{dE_\alpha} = -\frac{z^2e^4}{4\pi\epsilon_0m_e} \frac{nZ}{v^2} \ln\left(\frac{2m_ev^2}{I}\right) \symup{dx}
\end{equation}
Hierbei ist $z$ die Ladung und $v$ die Geschwindigkeit der $\alpha$-Teilchen, $I$ die Ionisierungsenergie, $Z$ die Ordnungszahl und $n$ die Teilchendichte der Luft.
Somit ist dieser Effekt direkt abhängig von der Energie der Strahlung und der Dichte der Luft.
Es lässt sich zudem eine Reichweite $R$ definieren:
\begin{equation}
	\label{R}
	R=-\int_0^{E_\alpha}\frac{\symup{dE_\alpha}}{\frac{\symup{dE_\alpha}}{dx}}
\end{equation}
\\
Für nierdige Energien verliert die Gleichung \eqref{bbgl} jedoch an Geltung. Deswegen wird aus empirisch gewonnenen Messergebnissen eine Gleichung für eine mittlere Reichweite $R_\text m$ bestimmt.
Die mittlere Reichweite wird definiert als den Abstand, bei welchem noch die Hälfte der $\alpha$-Teilchen vorhanden sind.
Es ergibt sich für $\alpha$-Strahlung in Luft mit Energien $E_\alpha \le \SI{2.5}{\mega\electronvolt}$:
\begin{equation}
	\label{Rm}
	R_\text m = E_\alpha^{3/2} \frac{\SI{3.1}{\centi\meter}}{\si{\mega\electronvolt}^{3/2}}
\end{equation}
Diese mittlere Reichweite ist in Gasen für konstantes Volumen und konstante Temperatur proportional zum Druck $p$.
Es lässt sich somit bei einer Messung mit variierendem Druck ein Rückschluss auf die Reichweite ziehen.
Hierfür ist es sinnvoll eine effektive Länge $x$ zwischen Quelle und Detektor zu definieren mit
\begin{equation}
	\label{xeff}
	x = x_0\frac{p}{p_0},
\end{equation}
wobei $x_0$ der tatsächliche Abstand zwischen Strahler und Detektor und $p_0 = \SI{1013}{\milli\bar}$\cite{v701} der Atmosphärendruck ist.
