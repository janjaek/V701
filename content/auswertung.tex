\section{Auswertung}
\label{sec:Auswertung}

Es werden zwei Messungen für zwei verschiedene feste Abstände durchgeführt.
Es werden Zählrate, Drücke und Channel aufgenommen.
Der Channel für die Messung bei $0\si{\milli\bar}$ entspricht $4\si{\mega\electronvolt}$.
Die effektive Länge wird mit Gleichung bestimmt.
Die so erhaltenen Werte sind in Tabelle\ref{tab:mess1} aufgetragen.
\begin{table}[H]
    \caption{Messwerte für einen festen Abstand von $x_0=2\si{\centi\meter}$.}
    \label{tab:mess1}
    \centering
    \begin{tabular}{S[table-format=4.0] S[table-format=1.2(0)e0] S[table-format=4.0(0)e0] S[table-format=2.2(0)e0]  }
        \toprule
        {Zählrate$1/120s$} & {Energie$/\si{\mega\electronvolt}$} & {Druck$/\si{\milli\bar}$} &{effektive Länge$\si{\centi\meter}$} \\
        \midrule
        74271 & 4.00 & 0 & 0.00\\
        73634 & 3.97 & 50 & 0.10\\
        71805 & 3.87 & 100 & 0.20\\
        69331 & 3.78 & 150 & 0.30\\
        67027 & 3.60 & 200 & 0.39\\
        64835 & 3.55 & 250 & 0.49\\
        61440 & 3.43 & 300 & 0.59\\
        33760 & 3.06 & 350 & 0.69\\
        28770 & 3.02 & 400 & 0.79\\
        41401 & 3.06 & 450 & 0.89\\
        41124 & 3.05 & 500 & 0.99\\
        33198 & 3.01 & 550 & 1.09\\
        27967 & 3.02 & 600 & 1.18\\
        20681 & 3.00 & 650 & 1.28\\
        15966 & 2.99 & 700 & 1.38\\
        9472 & 2.99 & 750 & 1.48\\
        6076 & 2.99 & 800 & 1.58\\
        2561 & 3.01 & 850 & 1.68\\
        937 & 2.99 & 900 & 1.78\\
        397 & 2.98 & 950 & 1.88\\
        99 & 3.00 & 1000 & 1.97\\

        \bottomrule
    \end{tabular}
\end{table}
\noindent Die Zählrate wird in Abbildung\ref{fig:c1} gegen die effektive Länge aufgetragen.
Es wird eine Reggressionsgerade mit SciPy/Python nach der Funktionsvorschrift
\begin{equation*}
  f(x) = mx +b
\end{equation*}
erstellt.
\begin{figure}[H]
  \centering
  \includegraphics{build/counts1.pdf}
  \caption{Lineare Reggression für die Zählrate in Abhängigkeit der effektiven Länge.}
  \label{fig:c1}
\end{figure}
\noindent Dabei ergeben sich folgenede Parameter:
\begin{equation*}
  m =(-384\pm29)\si{\per\centi\meter}
\end{equation*}
und
\begin{equation*}
  b =667\pm27  .
\end{equation*}
Die mittlere Reichweite ergibt sich dann als Schnittpunkt der Regressionsgeraden mit der Horizontalen bei $y=309.46$.
Die mittlere Reichweite kann, dann durch umstellen der Funktionsvorschrift, mit
\begin{equation*}
  R_m=\frac{309.46-b}{m}
\end{equation*}
bestimmt werden.
Der Fehler der mittleren Reichweite wird mit Gaußscher Fehlerfortpflanzung nach
\begin{equation*}
  \sigma_R = \sqrt{(\frac{b-309.46}{m^2}\sigma_m)^2 +(-\frac{1}{m}\sigma_b)^2}
\end{equation*}
berrechnet.
Der so erhaltene Wert beträgt $R_m= (0.93\pm0.1)\si{\centi\meter}$.
Damit lässt sich nach Gleichung eine Energie von $E_{\alpha} = 0.45\si{\mega\electronvolt}$ bestimmen.
Die Energie wird in Abbildung\ref{fig:e1} als Funktion der effektiven Länge aufgetragen.
Es wird eine Reggressionsgerade nach der Vorschrift
\begin{equation*}
  f(x) = mx + b
\end{equation*}
durch die Messwerte gezogen.
Für die Reggressionsrechnung wird nur der lineare Anteil der Messwerte verwendet.
\begin{figure}[H]
  \centering
  \includegraphics{build/energie1.pdf}
  \caption{Lineare Reggression für die Energie in Abhängigkeit der effektiven Länge.}
  \label{fig:e1}
\end{figure}
\noindent Der Energieverlust $-\frac{dE}{dx}$ kann dann als Steigung der Regression bestimmt werden.
Damit ergibt sich ein Energieverlust von
\begin{equation*}
  -\frac{dE}{dx} =(-0.52\pm0.07)\si{\mega\electronvolt\per\centi\meter} .
\end{equation*}
Analog wird für den zweiten Abstand vorgegangen.
Die Messwerte sind in Tabelle \ref{tab:mess2} aufgetragen
\begin{table}[H]
    \caption{Messwerte für einen festen Abstand von $x_0=1.5\si{\centi\meter}$.}
    \label{tab:mess2}
    \centering
    \begin{tabular}{S[table-format=4.0] S[table-format=1.2(0)e0] S[table-format=4.0(0)e0] S[table-format=2.2(0)e0]  }
        \toprule
        {Zählrate$1/120s$} & {Energie$/\si{\mega\electronvolt}$} & {Druck$/\si{\milli\bar}$} &{effektive Länge$\si{\centi\meter}$} \\
        \midrule
        148139 & 4.00 & 0 & 0.00\\
        150632 & 3.93 & 50 & 0.07\\
        149256 & 3.85 & 100 & 0.15\\
        143513 & 3.56 & 150 & 0.22\\
        139419 & 3.46 & 200 & 0.30\\
        145173 & 3.56 & 250 & 0.37\\
        136922 & 3.29 & 300 & 0.44\\
        134960 & 3.20 & 350 & 0.52\\
        133809 & 3.15 & 400 & 0.59\\
        138624 & 3.22 & 450 & 0.66\\
        138244 & 3.19 & 500 & 0.74\\
        130125 & 2.95 & 550 & 0.81\\
        131421 & 2.95 & 600 & 0.89\\
        125958 & 2.79 & 650 & 0.96\\
        123717 & 2.72 & 700 & 1.04\\
        121739 & 2.67 & 750 & 1.11\\
        120083 & 2.59 & 800 & 1.18\\
        117426 & 2.49 & 850 & 1.26\\
        114320 & 2.45 & 900 & 1.33\\
        112018 & 2.42 & 950 & 1.41\\
        104394 & 2.36 & 1000 & 1.48\\
        \bottomrule
    \end{tabular}
\end{table}
