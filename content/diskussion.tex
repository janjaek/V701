\section{Diskussion}
\label{sec:Diskussion}
Werden die Messwerte für die Zwei unterschiedlichen Entfernungen des $\alpha$-Strahler verglichen, so fällt auf, dass es nur schwer möglich ist den linearen Abfall der Zählrate und den linearen Abfall der Energie in einer Messung darzustellen.
Die Energiewerte für den ersten gemessenen Abstand nähern sich nach dem siebten Wert einem Plateau an.
Die Aussagekraft dieser Messung über den Energieverlust mit zunehmenden Druck ist dadurch stark gemindert.
Analog dazu kann für den zweiten Abstand eine große Streuung für die Zählrate afgetragen gegen die effektive Länge beobachtet werden.
Der lineare Abfall der Energie kann dafür besser betrachtet werden.
Wird die Statistik des radioaktiven Zerfalls betrachtet, kann aus den Histogrammen entnommen werden, dass eine Gaußverteilung schlecht die Statistik der Messwerte beschreibt.
Die Poissonverteilung nähert sich der tatsächlichen Verteilung der Messwerte gut an.
